\PassOptionsToPackage{unicode}{hyperref}
\PassOptionsToPackage{hyphens}{url}
\PassOptionsToPackage{table}{xcolor}
\PassOptionsToPackage{centerlast}{caption}
% \PassOptionsToPackage{chapter}{minted}

\documentclass[a4paper,12pt]{article}

% essential packages
% NB: loading mathtools after unicode-math seems to break underbracket
\usepackage{fontspec}
\usepackage[italian]{babel}
\usepackage{mathtools,amssymb,unicode-math}
\setmainfont{Times New Roman}
\setmathfont[Scale=MatchLowercase]{Cambria Math}
\setmonofont[Scale=MatchLowercase]{Iosevka}

% geometry
\usepackage[a4paper]{geometry}
\geometry{hmargin=1cm,vmargin=2cm}

% general packages
\usepackage[final]{microtype}
\usepackage{indentfirst}

\usepackage{float,flafter,caption,subcaption,graphicx,import}
\usepackage{xcolor,xurl}
\usepackage{qrcode}
\graphicspath{{assets/}}

\usepackage{multicol}

\usepackage{enumitem}
\setlist{noitemsep,topsep=0pt}
\setenumerate[1]{label={\arabic*)},ref={(\arabic*)}}
\setenumerate[2]{label={\arabic*.},ref={\arabic*.}}
\setdescription{wide, font=\normalfont}
\newlist{romenumerate}{enumerate}{2}
\setlist[romenumerate,1]{label={\Roman*)},ref={\Roman*}}
\setlist[romenumerate,2]{label={\roman*)},ref={\roman*}}
\newlist{hookdesc}{itemize}{1}
\setlist[hookdesc]{label=$\hookrightarrow$, left=\parindent, beginpenalty=10000}

% Space above and below a float in the middle of the main text
\setlength{\intextsep}{0.25cm}

% float parameters from https://texfaq.org/FAQ-floats
\renewcommand{\topfraction}{.85}
\renewcommand{\bottomfraction}{.7}
\renewcommand{\textfraction}{.15}
\renewcommand{\floatpagefraction}{.66}
\renewcommand{\dbltopfraction}{.66}
\renewcommand{\dblfloatpagefraction}{.66}
\setcounter{topnumber}{9}
\setcounter{bottomnumber}{9}
\setcounter{totalnumber}{20}
\setcounter{dbltopnumber}{9}

% math packages
\usepackage{booktabs,tabularx}
\usepackage{nicematrix}
\usepackage[makeroom]{cancel}
\renewcommand\CancelColor{\color{cadmiumred}}

\usepackage{witharrows}
\WithArrowsOptions{displaystyle,wrap-lines,tikz={-Latex, font={\small}}}
\AtBeginEnvironment{DispWithArrows*}{\ifvmode\noindent\smallskip\fi} % prevent overfull hbox

\NiceMatrixOptions{cell-space-limits = 2pt}
\newenvironment{NiceCases}{\left\lbrace\begin{NiceArray}{*{2}{>{\displaystyle}l}}}{\end{NiceArray}\right.}

\NewDocumentEnvironment{quantities}{m}%
    {\begin{NiceArray}{@{}*{#1}{>{\displaystyle}c@{\,\,\,\,}} @{\!\!\!\!}}}%
    {\end{NiceArray}}

\NewDocumentCommand\mathrnote{m}{\mathrlap{\qquad #1}}
\NewDocumentCommand\con{m}{\mathrnote{\text{con } #1}}

\NewDocumentCommand\hmath{m}{\renewcommand\fboxsep{0pt}\colorbox{cadmiumyellow}{#1}}

% tikz
\usepackage{tikz,tkz-base,tkz-euclide,circuitikz}
\usetikzlibrary{external,positioning,arrows.meta,shapes.geometric,matrix,fit}
\ctikzset{voltage/american minus={}, american voltages, american inductors}
% https://tex.stackexchange.com/a/570721/260531
\tikzexternalize[prefix=figures/]
\tikzexternaldisable
\NewDocumentCommand\ExternalizeThis{m}{\tikzexternalenable\tikzsetnextfilename{#1}}

\tikzset{
% https://tex.stackexchange.com/a/69225/260531
mid arrow/.style={postaction={decorate,decoration={
    markings,
    mark=at position .5 with {\arrow[#1]{Latex}}
}}},
point/.style={inner sep = 1pt, circle, fill},
}

\newlength\CircleArrowLength
\NewDocumentCommand\CircleArrowRight{mm}{
    \setlength\CircleArrowLength{\maxof{\widthof{#2}}{\heightof{#2}}}
    \draw [-Latex, anchor=center, shift=#1]
        node {#2}
        (0,1.2\CircleArrowLength) arc (85:-240:1.2\CircleArrowLength);
}
\NewDocumentCommand\CircleArrowLeft{mm}{
    \setlength\CircleArrowLength{\maxof{\widthof{#2}}{\heightof{#2}}}
    \draw [Latex-, anchor=center, shift=#1]
        node {#2}
        (0,1.2\CircleArrowLength) arc (85:-240:1.2\CircleArrowLength);
}

% table of contents
\usepackage{sectsty,tocloft}
\setcounter{tocdepth}{3}
\allsectionsfont{\bfseries}
\addtolength\cftsecnumwidth{1em}
\addtolength\cftsubsecnumwidth{1em}
\def\cftchapfont{\bfseries}
\renewcommand\cftsecfont{\bfseries}
\renewcommand\cftsubsecfont{\bfseries}
\renewcommand\cftsubsubsecfont{\bfseries}
\addto\captionsitalian{%
    \renewcommand\contentsname{\bfseries Indice}%
    \renewcommand\abstractname{\bfseries Abstract}%
}

% math commands
\usepackage{dsfont}
\newcommand\set[1]{\mathds{#1}}
\newcommand\R{\set{R}}
\newcommand\pinfty{{+\infty}}
\newcommand\minfty{{-\infty}}
\newcommand\seg[1]{\overline{#1}} % segment

\usepackage{derivative}
\NewCommandCopy\pdfrac\pdv

\DeclarePairedDelimiter\pars{\lparen}{\rparen}
\DeclarePairedDelimiter\brak{\lbrack}{\rbrack}
\DeclarePairedDelimiter\brac{\lbrace}{\rbrace}
\DeclarePairedDelimiter\angl{\langle}{\rangle}
\DeclarePairedDelimiter\bars{\lvert}{\rvert}
\DeclarePairedDelimiter\modu{\lVert}{\rVert}

\newcommand\quot[1]{``#1''}

\NewCommandCopy\mono\texttt
\newcommand\code[1]{\colorbox{gray!40}{\textcolor{cadmiumgreen}{\texttt{#1}}}}

\newcommand\vv{\vec{v}}
\newcommand\vu{\vec{u}}
\newcommand\vx{\vec{x}}
\newcommand\vw{\vec{w}}
\newcommand\va{\vec{a}}
\newcommand\vb{\vec{b}}
\newcommand\vzero{\vec{0}}
\newcommand\vr{\vec{r}}
\newcommand\vR{\vec{R}}
\newcommand\vM{\vec{M}}

\newcommand\hv{\hat{v}}
\newcommand\hu{\hat{u}}
\newcommand\he{\hat{e}}
\newcommand\hk{\hat{k}}

\NewDocumentCommand\f{s O{f} m}{#2\IfBooleanTF{#1}{\pars*}{\pars}{#3}}
\newcommand\finv[2][f]{#1^{-1}\pars{#2}}
\newcommand\fp[2][f]{\f[#1']{#2}}
\newcommand\fpp[2][f]{\f[#1'']{#2}}
\NewDocumentCommand\ft{m}{\f[#1]{t}}

\newcommand\g[2][g]{#1\pars{#2}}
\newcommand\gp[2][g]{\f[#1']{#2}}

\newcommand\iu{\mathrm{i}\mkern1mu}
\newcommand\notimplies{\nRightarrow}

\newcommand\md{\mathrm{d}}
\newcommand\dx{\md x}
\newcommand\dy{\md y}
\newcommand\dz{\md z}
\newcommand\dt{\md t}
\newcommand\Dt{\Delta t}

\ifdefined\parallelslant\renewcommand\parallel\parallelslant\fi

\newcommand\qed{\hfill\ensuremath{\square}}

\newcommand\rg{\mathop{\mathrm{rg}}\nolimits}
\newcommand\im{\mathop{\mathrm{Im}}\nolimits}
\renewcommand\hom[2]{\mathop{\mathrm{Hom}}\nolimits\pars{#1, #2}}
\newcommand\sgn{\mathop{\mathrm{sgn}}\nolimits}
\newcommand\trans[1]{#1^{\mathrm{T}}}

\newcommand\existsunique{\mathop{\exists\!!}}

% https://tex.stackexchange.com/a/520268
\newenvironment{system}{\begin{cases}}{\end{cases}}
\NewDocumentEnvironment{LabeledSystem}{O{}}{%
    \begin{NiceArray}{@{}r<{\colon} @{\;\;} \{ >{\displaystyle}l #1 @{}}}{%
    \end{NiceArray}}

\newcommand\hquad{\hspace{0.5em}}

% siunitx
\usepackage{siunitx}
\sisetup{per-mode=symbol, exponent-product=\cdot, inter-unit-product=\cdot}

% text commands
\usepackage{emoji}
\setemojifont{Twemoji Mozilla}
\NewDocumentCommand\sparkle{m}{\emoji{sparkles}#1\emoji{sparkles}}

% colors
\definecolor{darkgreen}{rgb}{0.0, 0.5, 0.0}
\definecolor{chromeyellow}{rgb}{1.0, 0.65, 0.0}
\definecolor{aquamarine}{rgb}{0.5, 1.0, 0.83}
\definecolor{airforceblue}{rgb}{0.36, 0.54, 0.66}
\definecolor{bostonuniversityred}{rgb}{0.8, 0.0, 0.0}
\definecolor{brightgreen}{rgb}{0.4, 1.0, 0.0}
\definecolor{burgundy}{rgb}{0.5, 0.0, 0.13}
\definecolor{cadmiumred}{rgb}{0.89, 0.0, 0.13}
\definecolor{cadmiumgreen}{rgb}{0.0, 0.42, 0.24}
\definecolor{cadmiumorange}{rgb}{0.93, 0.53, 0.18}
\definecolor{cadmiumyellow}{rgb}{1.0, 0.96, 0.0}
\definecolor{darkturquoise}{rgb}{0.0, 0.81, 0.82}
\definecolor{darkviolet}{rgb}{0.58, 0.0, 0.83}

\colorlet{shade}{gray!40}

% quotations
\usepackage{csquotes}
\renewcommand\mkcitation[1]{\par\hfill #1}
\renewcommand\mkccitation[1]{\par\hfill #1}
\SetBlockEnvironment{quotation}

% bibliography
\NewDocumentCommand\SetupBibliography{m}{
    \usepackage[backend=biber, style=numeric]{biblatex}
    \addbibresource{#1}
}

% footnotes
\renewcommand\thefootnote{\Roman{footnote}}

% header
\usepackage{titling,fancyhdr}
\pagestyle{empty}
% \pagestyle{fancy}
% \DeclareDocumentCommand\chaptermark[1]{\markboth{[\thechapter]\ #1}{}}
% \fancyhead{}
% \fancyfoot{}
% \fancyhead[LO,RE]{\bfseries\thepage}
% \fancyhead[C]{\thetitle}
% \fancyhead[LE,RO]{\bfseries\today}
% \fancyfoot[C]{\theauthor}
% \fancyfoot[RO,LE]{\bfseries\leftmark}
% \fancyfoot[C]{\bfseries:)}
% \fancypagestyle{plain}{}

% load package bookmark & hyperref last as per the instructions
\usepackage{bookmark}
\hypersetup{
    colorlinks,
    linkcolor={burgundy},
    citecolor={cadmiumgreen},
    urlcolor={darkturquoise}
}

\AtBeginDocument{%
    % display skip is reset upon \begin{document} so we must set them this way
    \addtolength\belowdisplayshortskip{-3pt}%
    \addtolength\abovedisplayshortskip{-3pt}%
    \addtolength\abovedisplayskip{-5pt}%
    \addtolength\belowdisplayskip{-5pt}%
    \addtolength\belowcaptionskip{3pt}%
}

% subfigure with automatic width
% adapted from https://www.reddit.com/r/LaTeX/comments/42suro/automatically_adjusting_width_in_subfigure/czdsgh8/
\newlength\figwidth
\newsavebox\figbox
\NewDocumentCommand\AutoSubcaption{s O{0pt} m m}{%
    \savebox{\figbox}{#4}%
    \settowidth{\figwidth}{\usebox{\figbox}}%
    \addtolength{\figwidth}{#2}%
    \IfBooleanTF#1%
        {\begin{subtable}[t]{\figwidth}\centering\caption{#3}\usebox{\figbox}\end{subtable}}%
        {\begin{subfigure}[t]{\figwidth}\centering\caption{#3}\usebox{\figbox}\end{subfigure}}%
}

% show date of lecture in margin
\usepackage{marginnote}
\renewcommand*\marginfont{\footnotesize\color{darkviolet}}
\NewDocumentCommand\lecturemark{m}{\protect\marginnote{#1}}

% add a non-numbered chapter with correct toc handling
\NewDocumentCommand\UnnumberedChapter{m}{%
\chapter*{#1}%
\markboth{#1}{#1}%
\addcontentsline{toc}{chapter}{#1}%
}

\NewDocumentCommand\SetupMinted{}{
    % \setminted{autogobble,breaklines,breakbytokenanywhere}
    % \usemintedstyle{molokai} % requires pip package pygments-molokai
    % \definecolor{codeBg}{HTML}{1c1c1e}
    % \setminted{bgcolor=codeBg}
    \usepackage{tcolorbox}
    \tcbuselibrary{minted}
    \tcbuselibrary{breakable}
    \tcbset{listing only,breakable,fonttitle=\bfseries,float*=hbtp,width=\textwidth}
}


\SetupMinted{}
\SetupBibliography{relazione.bib}
\pagestyle{empty}

\begin{document}

\twocolumn[{\begin{figure}[H]
    \setlength{\linewidth}{\textwidth}
    \setlength{\hsize}{\textwidth}
    \centering
    \begin{NiceTabular}{rl}
            \Block{3-1}{\includegraphics[height=20ex]{logo}} & {\Large\bfseries Seconda relazione di fattibilità: Carroponte smart}                                        \\
                                                             & Giuseppe Damiata, Vincenzo Mazza, Nicolò Spingola \\
                                                             & Università degli studi di Palermo, Dipartimento di Ingegneria, Hangar\_9
                                                             \\ \\ \\ % extra rows for logo spacing
    \end{NiceTabular}
\end{figure}}]

\begin{abstract}
    Nella presente relazione di fattibilità si propone un progetto per un carroponte smart, capace
    di muovere automaticamente dei container all'interno di un porto. Per la realizzazione di tale
    progetto si intende utilizzare in combinazione un PLC Zelio Smart Relay SR3B261BD ed un Arduino
    MEGA, al fine di dimostrare la capacità di sistemi ad-hoc di essere integrati a posteriori in
    una rete robusta di automazione industriale.
\end{abstract}

\section{Introduzione}

Come anticipato nell'abstract, il progetto che si propone ha come obiettivo la realizzazione di un
carroponte smart, ovvero un carroponte dotato di un sistema di controllo che gli permetta di
realizzare in totale autonomia carichi e scarichi di container da una nave merci. \cite{ren2021design,gupta2004simplified}

In particolare, il sistema è realizzato mediante la sinergia di due dispositivi: Un PLC Zelio
SR3B261BD ed un Arduino MEGA. Tale scelta è stata fatta per dimostrare la capacità di sistemi
precedentemente realizzati ad-hoc, fini ad un solo controllo automatico locale, di essere integrati
grazie al protocollo Modbus nella vasta rete di automazione industriale di un azienda di trasporti,
grazie a sistemi SCADA (\emph{Supervisory Control and Data Acquisition}). \cite{li2015data,arm2015real}

\section{Descrizione del sistema}

Il sistema consiste delle seguenti parti.
\begin{itemize}
    \item Un microcontrollore Arduino MEGA, che si interfaccia con un apposito connettore al PLC
        Zelio SR3B261BD, e che è in grado di controllare i servomotori del carroponte, abilitare e
        disabilitare l'elettromagnete, e leggere i fotoresistori che indicano la presenza o assenza
        di container da spostare.
    \item Un PLC Zelio SR3B261BD, che si interfaccia appunto con l'Arduino MEGA, e che è in grado di
        monitorare i dati ad esso forniti dal microcontrollore e manipolare ogni aspetto di
        quest'ultimo, al fine di consentire il telecontrollo e l'acquisizione di dati.
    \item Tre servomotori che permettono di muovere il carroponte lungo le tre direzioni
        degli assi cartesiani.
    \item Sei fotoresistenze corrispondenti alle sei aree di carico/scarico.
    \item Un elettromagnete, controllato in corrente continua, che permette di sollevare e trasportare i container.
\end{itemize}

Il PLC Zelio viene dotato di un modulo di espansione SR3NET01BD, che permette di interfacciarsi con
il sistema SCADA mediante il protocollo Modbus TCP/IP. Il sistema SCADA consiste poi in due
programmi, uno realizzato in Vijeo Citect per utilizzo desktop e l'altro in TeslaSCADA per utilizzo
mobile.

\begin{figure*}[htbp]\centering
    \caption{Lo schema elettrico del sistema, realizzato in KiCAD.}
    \includegraphics[width=\linewidth]{schema-elettrico.pdf}
\end{figure*}

\section{Benefici del sistema}

In conclusione, il sistema proposto permette di aumentare l'efficienza nonché la sicurezza di un
tradizionale carroponte, o addirittura anche di uno già automatizzato ma privo di SCADA, senza
cambiamenti troppo costosi all'architettura di un sistema già esistente. Vengono ridotti
notevolmente i tempi di fermo macchina, nonché le possibilità di errori umani o pericolose cadute
dovute ad arrampicamenti sul carroponte.

\enlargethispage*{4\baselineskip}

\printbibliography

\end{document}
% vim: ft=tex
